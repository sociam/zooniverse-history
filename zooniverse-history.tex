\documentclass{sigchi}

% Use this command to override the default ACM copyright statement (e.g. for preprints). 
% Consult the conference website for the camera-ready copyright statement.
\toappear{}

% Arabic page numbers for submission. 
% Remove this line to eliminate page numbers for the camera ready copy
%\pagenumbering{arabic}


% Load basic packages
\usepackage{balance}  % to better equalize the last page
\usepackage{graphics} % for EPS, load graphicx instead
\usepackage{times}    % comment if you want LaTeX's default font
\usepackage{url}      % llt: nicely formatted URLs
\usepackage{algorithm,algorithmic}

% llt: Define a global style for URLs, rather that the default one
\makeatletter
\def\url@leostyle{%
  \@ifundefined{selectfont}{\def\UrlFont{\sf}}{\def\UrlFont{\small\bf\ttfamily}}}
\makeatother
\urlstyle{leo}


% To make various LaTeX processors do the right thing with page size.
\def\pprw{8.5in}
\def\pprh{11in}
\special{papersize=\pprw,\pprh}
\setlength{\paperwidth}{\pprw}
\setlength{\paperheight}{\pprh}
\setlength{\pdfpagewidth}{\pprw}
\setlength{\pdfpageheight}{\pprh}

% Make sure hyperref comes last of your loaded packages, 
% to give it a fighting chance of not being over-written, 
% since its job is to redefine many LaTeX commands.
\usepackage[pdftex]{hyperref}
\hypersetup{
pdftitle={SIGCHI Conference Proceedings Format},
pdfauthor={LaTeX},
pdfkeywords={SIGCHI, proceedings, archival format},
bookmarksnumbered,
pdfstartview={FitH},
colorlinks,
citecolor=black,
filecolor=black,
linkcolor=black,
urlcolor=black,
breaklinks=true,
}

% create a shortcut to typeset table headings
\newcommand\tabhead[1]{\small\textbf{#1}}

% End of preamble. Here it comes the document.
\begin{document}

\title{One Does Not `Simply' Launch a Citizen Science Project: Reflections on Zooniverse, a Multi-Domain Science Platform}
% \title{A Brief History of Zooniverse: Designing for Multi-Domain Citizen Science}

\numberofauthors{1} \author{ (Authors removed for reviewing) }
\maketitle

\begin{abstract}

\end{abstract}

\keywords{Citizen science, crowdsourcing, interface design}

%% TODO 
\category{H.5.m.}{Information Interfaces and Presentation (e.g. HCI)}{Miscellaneous}

%% TODO 
%% \terms{}

\section{Introduction}

Web-based ``citizen-science'' projects have enabled tens of thousands
of untrained human volunteers to contribute to open scientific
problems across a variety of domains.  The handful of successful
systems have demonstrated that citizen science applications can be
valuable both to participants, as educational tools and
cognitively-stimulating puzzles, and to scientific researchers with
large data sets and complex problem spaces.

However, designing successful citizen science systems that can be
mutually beneficial in this way while sustaining participation over a
time can be exceedingly difficult.  The reasons are several: first,
%needs to be a second somewhere
due to the emerging nature of the field, \emph{best practices} are not
yet known.  Unlike types of human computation that apply extrinsic
motivation (e.g., direct rewards) to drive participation, most citizen
science involvement derives from the many, varied intrinsic
incentives of individuals.  Since such motivations are typically
personal and idiosyncratic, designing to drive sustained attention to
them is challenging, requiring an understanding of their nature and
the activities that inspire them.  Participants naturally engage with
citizen science systems in a variety of different ways, and feature a
diversity of natural competencies, which is manifests itself in a wide
variation in the ways people perform tasks and activities in the
system. Finally, the simple fact that, over time, people are likely to
simply forget about any particular system, as their attention is
turned away towards newer things (activities and systems) that successfully compete,
 if only for their novelty.

In this paper, we contribute a detailed case-study of a unique
citizen-science platform. This expanded from a single domain
experiment into a general, open platform for launching and hosting
citizen science projects, known as Zooniverse.  This analysis,
conducted as a retrospective reflective interview study over its five
years of evolution, was conducted with key members of the Zooniverse
team.  The focus of this reflection was toward identifying the key
design and deployment observations derived largely from experimental
trial and error; spanning  of the results of  decisions
across all aspects of the project's design, from interface, to
discussion forums, to outreach and

the  led to success in some
prom findings, and iteration across 5 key aspects of designing the
system,

The Zooniverse framework team has derived significant has
been successively refined and scaled as the variety of tasks and
number of participants have increased.  At its current state,
currently having launched $X$ distinct applications for $Y$ scientific
domains, including astronomy, zoology, cell and marine biology,
archaeology and paleontology.  This platform represents a unique\cite{moore2011facebooking}


%%  These
%% applications, though separate, have been built on top 

%% The experiences from the first were used to derive design goals for
%% the next,

%% The contributions of the 
%% We identify key design challenges

%% especially as the best practices for designing citizen science systems
%% has not yet emerged.  Among the many design challenges include, being
%% able to appeal to participants with an extremely wide range of
%% expertise, ranging from no knowledge of the field to significant
%% background and interest.  Participants naturally feature a diversity
%% of natural competencies, which is manifested in some people being
%% simply much more adept at some tasks than others. Second, people have
%% many different reasons for engaging with citizen science projects, and
%% to sustain engagement, these platforms must appeal to, and engage
%% these different motivating reasons. Finally, there are a large variety
%% of issues pertaining to individual retention, well as supporting
%% various degrees of engagement -- from the ``sunday scientist'' to the
%% ``scienceoholic''.


%% The purpose of this examination of Zooniverse is to both to document
%% the experience gained from launches and iterations of the various
%% applications, comparing these experiences against previously
%% documented in other citizen-science projects.  The observations derive
%% from a lateral examination of the

%% The path from its first experimental app, Galaxy Zoo, to the more than
%% twenty different projects that have launched on the Zooniverse project
%% required generalising the findings from the first project to different
%% kinds of tasks in other scientific domains.

%%  naturally Participants come from a wide
%% audience % with a massive variety of backgrounds and competencies,
%% such systems interface down to the workflow of how participants' input
%% is collated, verified, and provided as feedback to the participants,
%% along with the nature and kind(s) of affordances provided for
%% communicating and discussing remains challenigng

%% interfaces that have
%% appropriate affordances, the and features remains challenging, due
%% to the wide number of design considerations that mustbe taken
%% jointly into account.

%% Wide variety of expertise

% \section{Background: Brief History of Zooniverse}

% \emph{For the CSCW readers, outline the history of the development of the system
% including a detailed description}

% \section{Observations through iterations}

% \emph{I was thinking put key design observations here relating to how to cross-domain
% citizen science}

% \section{$D$ Myths of Designing for Citizen-Science}

% \emph{I was thinking put key design observations here}

% \subsection{Myth $X$: Putting new users through a ``tutorial'' is a good idea}

% \subsection{Myth $Y$: Gameification keeps people motivated}

% \subsection{Myth $Z$: Participants become domain experts}


\section{Related work}

\emph{Connect related work here with FoldIt, etc}

\section{Discussion}

\section{Conclusion}

\section{Acknowledgments}
Acknowledgments omitted for blind review.

\balance

% If you want to use smaller typesetting for the reference list,
% uncomment the following line:
% \small
\bibliographystyle{acm-sigchi}
\bibliography{zooniverse-history}
\end{document}

%% from crw04
%% \begin{algorithm}[tb]
%%   \caption{Overview of our general negotiation process, which is common to all of our strategies.  Let $o_\text{own}$ and $o_\text{opp}$ represent our own and the opponent's latest offers, respectively. $t_c$ is the current time and $u_\tau$ is the aspiration level at time $t_c$.}\label{alg:generic-overview}
%%   \begin{algorithmic}
%%     \FOR{$t_c \in [0,1]$}
%%     \STATE $o_\text{opp} \Leftarrow $ {\sc ReceiveOffer}()
%%     \STATE $u_\tau \Leftarrow $ {\sc SetAspirationLevel}($o_\text{opp}, t_c$)
%%     \IF{{\sc GetUtility}($o_\text{opp}, t_c$) $\geq u_\tau$}
%%     \STATE {\sc AcceptOffer}($o_\text{opp}$)
%%     \RETURN
%%     \ENDIF
%%     \STATE $o_\text{own} \Leftarrow $ {\sc GenerateOffer}($u_\tau$)
%%     \STATE {\sc ProposeOffer}($o_\text{own}$)
%%     \ENDFOR
%%     \end{algorithmic}
%% \end{algorithm}

%%  LocalWords:  artefacts HCI artefact Dropbox Skydrive Google PDF
%%  LocalWords:  LaTeX versioning throughs interactional CDSSes UI LD
%%  LocalWords:  bioinformaticians iPad iCloud iCal favour favourite
%%  LocalWords:  microformats picoformats WebDAV situ VCS scm priori
%%  LocalWords:  Powerpoint CB's CBs each's bulleted parseable OTs
%%  LocalWords:  sub-schemas pre Dourish XLSX csv PPTX PPT ICS CalDAV
%%  LocalWords:  RSS VCF XSLT XLST CSS Dojo PNG
