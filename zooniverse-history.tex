\documentclass{sigchi}

% Use this command to override the default ACM copyright statement (e.g. for preprints). 
% Consult the conference website for the camera-ready copyright statement.
\toappear{}

% Arabic page numbers for submission. 
% Remove this line to eliminate page numbers for the camera ready copy
%\pagenumbering{arabic}


% Load basic packages
\usepackage{balance}  % to better equalize the last page
\usepackage{graphics} % for EPS, load graphicx instead
\usepackage{times}    % comment if you want LaTeX's default font
\usepackage{url}      % llt: nicely formatted URLs
\usepackage{algorithm,algorithmic}

% llt: Define a global style for URLs, rather that the default one
\makeatletter
\def\url@leostyle{%
  \@ifundefined{selectfont}{\def\UrlFont{\sf}}{\def\UrlFont{\small\bf\ttfamily}}}
\makeatother
\urlstyle{leo}


% To make various LaTeX processors do the right thing with page size.
\def\pprw{8.5in}
\def\pprh{11in}
\special{papersize=\pprw,\pprh}
\setlength{\paperwidth}{\pprw}
\setlength{\paperheight}{\pprh}
\setlength{\pdfpagewidth}{\pprw}
\setlength{\pdfpageheight}{\pprh}

% Make sure hyperref comes last of your loaded packages, 
% to give it a fighting chance of not being over-written, 
% since its job is to redefine many LaTeX commands.
\usepackage[pdftex]{hyperref}
\hypersetup{
pdftitle={SIGCHI Conference Proceedings Format},
pdfauthor={LaTeX},
pdfkeywords={SIGCHI, proceedings, archival format},
bookmarksnumbered,
pdfstartview={FitH},
colorlinks,
citecolor=black,
filecolor=black,
linkcolor=black,
urlcolor=black,
breaklinks=true,
}

% create a shortcut to typeset table headings
\newcommand\tabhead[1]{\small\textbf{#1}}

% End of preamble. Here it comes the document.
\begin{document}

\title{A Brief History of Zooniverse: Designing for Multi-Domain Citizen Science}

\numberofauthors{1} \author{ (Authors removed for reviewing) }
\maketitle

\begin{abstract}

\end{abstract}

\keywords{Citizen science, crowdsourcing, interface design}

%% TODO 
\category{H.5.m.}{Information Interfaces and Presentation (e.g. HCI)}{Miscellaneous}

%% TODO 
%% \terms{}

\section{Introduction}

Citizen-science is a type of human-powered computation \cite{} that
can be both beneficial to those who participate, as both an
educational tool and cognitively-stimulating source of entertainment,
as well as an unprecedented method of faciliting the discovery of
significant, novel scientific findings. The result of the relatively
few citizen-science projects on the Web today have already generated a
comparably large number of findings \cite{}, through the collective
contributions of hundreds of thousands of volunteers.

However, designing effective science apps that can achieve both goals
can be challenging.  First, such systems must appeal to participants
with an extremely wide range of expertise, ranging from no knowledge
of the field to significant background and interest.  Moreover,
particpants naturally feature a diversity of natural competencies,
which is manifested in some people being simply much more adept at
some tasks than others. Finally, there are a large variety of issues
pertaining to keeping individuals motivated, interested, and deriving
personal benefit while participating, as well as supporting various
degrees of engagement -- from the ``sunday scientist'' to the
``scienceoholic''.

In this paper, we provide a detailed case study of a citizen-science
platform which offers the unique position of having expanded from one
experiment focused on a single domain to more than $X$ distinct
projects spanning $Y$ domains, including astronomy, zoology, marine
biology, archaeology, and paleontology, over its two year evolution.
These applications, though separate, have been built on top of a
single unified framework known as Zooniverse, which has been
successively refined and scaled as the variety of tasks and number of
participants have increased.

The purpose of this examination of Zooniverse is to both to document
the experience gained from launches and iterations of the various
applications, comparing these experiences against previously
documented in other citizen-science projects.  The observations derive
from a lateral examination of the

The path from its first experimental app, Galaxy Zoo, to the more than
twenty different projects that have launched on the Zooniverse project
required generalising the findings from the first project to different
kinds of tasks in other scientific domains.

%%  naturally Participants come from a wide
%% audience % with a massive variety of backgrounds and competencies,
%% such systems interface down to the workflow of how participants' input
%% is collated, verified, and provided as feedback to the participants,
%% along with the nature and kind(s) of affordances provided for
%% communicating and discussing remains challenigng

%% interfaces that have
%% appropriate affordances, the and features remains challenging, due
%% to the wide number of design considerations that mustbe taken
%% jointly into account.

%% Wide variety of expertise

\section{Brief History of Zooniverse}

\section{Key Challenges}

\section{Myths of Citizen Science Projects}

\subsection{Gameification keeps people motivated}

\subsection{Participants become domain experts}

\section{Discussion}

\section{Conclusion}

\section{Acknowledgments}
Acknowledgments omitted for blind review.

\balance

% If you want to use smaller typesetting for the reference list,
% uncomment the following line:
% \small
\bibliographystyle{acm-sigchi}
\bibliography{morph}
\end{document}

%% from crw04
%% \begin{algorithm}[tb]
%%   \caption{Overview of our general negotiation process, which is common to all of our strategies.  Let $o_\text{own}$ and $o_\text{opp}$ represent our own and the opponent's latest offers, respectively. $t_c$ is the current time and $u_\tau$ is the aspiration level at time $t_c$.}\label{alg:generic-overview}
%%   \begin{algorithmic}
%%     \FOR{$t_c \in [0,1]$}
%%     \STATE $o_\text{opp} \Leftarrow $ {\sc ReceiveOffer}()
%%     \STATE $u_\tau \Leftarrow $ {\sc SetAspirationLevel}($o_\text{opp}, t_c$)
%%     \IF{{\sc GetUtility}($o_\text{opp}, t_c$) $\geq u_\tau$}
%%     \STATE {\sc AcceptOffer}($o_\text{opp}$)
%%     \RETURN
%%     \ENDIF
%%     \STATE $o_\text{own} \Leftarrow $ {\sc GenerateOffer}($u_\tau$)
%%     \STATE {\sc ProposeOffer}($o_\text{own}$)
%%     \ENDFOR
%%     \end{algorithmic}
%% \end{algorithm}

%%  LocalWords:  artefacts HCI artefact Dropbox Skydrive Google PDF
%%  LocalWords:  LaTeX versioning throughs interactional CDSSes UI LD
%%  LocalWords:  bioinformaticians iPad iCloud iCal favour favourite
%%  LocalWords:  microformats picoformats WebDAV situ VCS scm priori
%%  LocalWords:  Powerpoint CB's CBs each's bulleted parseable OTs
%%  LocalWords:  sub-schemas pre Dourish XLSX csv PPTX PPT ICS CalDAV
%%  LocalWords:  RSS VCF XSLT XLST CSS Dojo PNG
