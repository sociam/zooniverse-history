\documentclass{sigchi}

% Use this command to override the default ACM copyright statement (e.g. for preprints). 
% Consult the conference website for the camera-ready copyright statement.
\toappear{}

% Arabic page numbers for submission. 
% Remove this line to eliminate page numbers for the camera ready copy
%\pagenumbering{arabic}


% Load basic packages
\usepackage{balance}  % to better equalize the last page
\usepackage{graphics} % for EPS, load graphicx instead
\usepackage{times}    % comment if you want LaTeX's default font
\usepackage{url}      % llt: nicely formatted URLs
\usepackage{algorithm,algorithmic}

% llt: Define a global style for URLs, rather that the default one
\makeatletter
\def\url@leostyle{%
  \@ifundefined{selectfont}{\def\UrlFont{\sf}}{\def\UrlFont{\small\bf\ttfamily}}}
\makeatother
\urlstyle{leo}


% To make various LaTeX processors do the right thing with page size.
\def\pprw{8.5in}
\def\pprh{11in}
\special{papersize=\pprw,\pprh}
\setlength{\paperwidth}{\pprw}
\setlength{\paperheight}{\pprh}
\setlength{\pdfpagewidth}{\pprw}
\setlength{\pdfpageheight}{\pprh}

% Make sure hyperref comes last of your loaded packages, 
% to give it a fighting chance of not being over-written, 
% since its job is to redefine many LaTeX commands.
\usepackage[pdftex]{hyperref}
\hypersetup{
pdftitle={SIGCHI Conference Proceedings Format},
pdfauthor={LaTeX},
pdfkeywords={SIGCHI, proceedings, archival format},
bookmarksnumbered,
pdfstartview={FitH},
colorlinks,
citecolor=black,
filecolor=black,
linkcolor=black,
urlcolor=black,
breaklinks=true,
}

% create a shortcut to typeset table headings
\newcommand\tabhead[1]{\small\textbf{#1}}

% End of preamble. Here it comes the document.
\begin{document}

\title{One Does Not `Simply' Launch a Citizen Science Project: Reflections on Zooniverse, a Multi-Domain Science Platform}
% \title{A Brief History of Zooniverse: Designing for Multi-Domain Citizen Science}

\numberofauthors{1} \author{ (Authors removed for reviewing) }
\maketitle

\begin{abstract}

\end{abstract}

\keywords{Citizen science, crowdsourcing, interface design}

%% TODO 
\category{H.5.m.}{Information Interfaces and Presentation (e.g. HCI)}{Miscellaneous}

%% TODO 
%% \terms{}

\section{Introduction}

Web-based ``citizen-science'' projects have enabled hundreds of
thousands of untrained human volunteers to contribute to open
scientific problems across a variety of domains
\cite{citizen-science}.  The handful of successful systems have
demonstrated that citizen science applications can be valuable both to
participants, as educational tools and cognitively-stimulating
puzzles\cite{citizen-science-in-curricula}, as well as to scientific
researchers who have large unmanageable data sets and problem spaces
\cite{fortson-2011, lintott-08, lintott-11, simpson-12, davis-11}.

% zooniverse et al publications >
% https://www.zooniverse.org/publications 
% http://mnras.oxfordjournals.org/content/424/4/2442.full

% davis-12 : The distribution of interplanetary dust between 0.96
% and 1.04 au as inferred from impacts on the STEREO spacecraft
% observed by the heliospheric imagers, Davis+ 2012.  

% sayighs : Repeated call types in short-finned pilot whales,
% Globicephala macrorhynchus, Sayigh+ 2012.

However, designing successful citizen science systems that can be
mutually beneficial in this way while sustaining participation over
time can be exceedingly difficult.  The reasons are several: first,
due to the emerging nature of the field, key design principles and
\emph{best practices} are not yet known or established.  The unique
aspects of designing for citizen science mean that conventional design
methods pioneered for human computation approaches only apply to a
limited extent.  For example, human computation systems often apply
extrinsic motivation (e.g., direct rewards) to drive participation,
while most citizen science participation systems derive from intrinsic
motivations of individuals \cite{extrinsic-vs-intrinsic}.  Since such
motivations are typically personal and idiosyncratic, designing to
drive sustained attention to them is challenging, requiring an
understanding of their nature and the activities that inspire them
\cite{}.  Moreover, participants naturally engage with citizen science
systems in a variety of different ways, and feature a diversity of
natural competencies, manifesting in great variation among the ways
people interact with the system \cite{raddick}. Finally, the simple
fact that, over time, people are likely to simply forget about any
particular system, as their attention is turned away towards other,
newer systems and activities means that retaining those crucial
participants who have accrued a degree of experience and competence in
their tasks is significantly challenging.

% we want to switch this from interviews to (in crowd). 
% TODO - Ramine - please update this ``retrospective reflective inductive...''
% to something that makes sense - we sat as a group with key team mebers and identified
% themes
In this paper, we contribute a detailed case-study of a unique
citizen-science platform which expanded from a single domain
experiment to what has been considered a ``factory for citizen
science'', an authority for generating successful web applications for
scientists with difficult problems independent of any particular
domain.  Our analysis, conducted as retrospective reflective inductive
thematic analysis with key Zooniverse team members, aimed to identify
the essential lessons learned through the trial and error process that
resulted from designing each app after the previous.  This iterative
process allowed the team to explore design variations at mutiple
levels, from the interface/intraction flow, to the design of social
elements (discussion forums), to external considerations such as ways
to best launch and sustain prolonged engagement through external
communications.

% note: add the fact that this is data analysis not data acquisition
The goal of the this paper is to, first, document the informal
knowledge gathered by the Zoonvierse team pertaining to how to design
successful citizen science projects, based on their experience of
launching 24 projects since the platform's genesis.  These insights
are then discussed in the greater context of human computation, to
derive design recommendations and discuss factors that may be
responsible for the observations made. We begin with a short history
of the project in order to provide readers a context for the following
discussion, followed by a grounded dimensional design analysis of
particular aspects of Zooniverse's deployment.  Finally, we discuss 
Zooniverse's team's perspective of the greatest difficulties for
building more effective citizen science: $X$, $Y$ and $Z$, and
ways that HCI research may be able to help.

\section{Background: A Brief History of the Zooniverse}

% \emph{TODO: Can someone fill this in?}  The Zooniverse system was conceived .. \cite{fortson2011galaxy} \emph{TODO TODO --- } 

In contrast to crowdsourcing and citizen science projects that focus on participant-driven data collection (e.g., \cite{okolloh2009ushahidi}, \cite{zook2010volunteered}), the focus of Zooniverse is exclusively citizen \emph{data analysis}, that is, to get participants to help classify, label and identify information in large, already extant sets. As documented previously by Fortson et al \cite{fortson2011galaxy}, the first Zooniverse project, Galaxy Zoo, was designed to engage volunteers in the morphological classification of images of galaxies, and launched in July 2007\cite{galaxyzoo-launch}.

\emph{AND THEN STUFF HAPPENED!} 

2008 - publication of ``Galaxy Zoo: morphologies'' paper (MNRAS)
2008 - publication of ``Galaxy Zoo: the large-scale spin statistics of spiral galaxies in the Sloan Digital Sky Survey''  paper (MNRAS) 
After Galaxy Zoo it was clear that more projects were to follow.
In develping GZ2 Arfon Smith created `Juggernaut' a Ruby on Rails application, with a MySQL back end, designed as a generalised web application for citizen science projects like Galaxy Zoo.
%% Neale this is when the ideas of annotations, classifications etc came into being
Early 2010 saw Solar Stormwatch happen, a project effectively built for someone else (Royal Observatory Greenwich) by the Zooniverse. It used Juggernaut and is still running.
<<When does the CSA happen!?>>
October 2010 sees the deployment of Old Weather - the first non-astronomy project, first transcription project and first to incorporate any game mechanics.
Around the same time we began to take social media more seriously and created a blog network, Twitter/Facebook accounts and even held special events such as the (now annual) Zooniverse Advent Calendar in Dec 2010.
The Milky Way Project and Planet Hunters launche din Dec 2010 both using public astronomy datasets (like Galaxy Zoo).
In July 2011 Anicent Lives launches - another non-astronomy, transcription project.
In August 2011 we created `Talk' as a way to begin to better harness the power of our discussion fora. The PHP fora we used beforehand were beocming difficult to support and hard to mine for data. We felt that there was a better user experience to be created and a better way for citizen scienceto happen, base don the experienc eon the Peas Corp and the stor yof the Voorwerp.
In August 2011 Ice Hunters went live. Built by a separate development team using our system.
In Oct 2011 NEEMO was our first purely experimental project - where we weren't even sure of the scientific outcome. We were very clear about this with the users and even housed it in a new `labs' area of the Zooniverse site to be clear that it was an experiment.
Whale FM (Nov 29 2011) was our first project to use sound. It was clear early on that it wasn't as popular as the others.
In January 2012, BBC Stargazing Live featured Planet Hunters and we received a million classiciations in just two and a bit days. Planet Hunters was actually more popular in the days following the announcement of our discovering a planet during that time, that it had been during the BBC show.
In early 2012 we came up with the idea of switching to MongoDB and HTML5 web-apps served from Amazon S3 buckets. The new system: `Ouroboros' would replace Juggernaut and act as a one app serving all our projects. The Juggernaut codebased had become fragmented and each project required its own server and DB. It was getting expensive! IN Sep 2012 we ran `Citizen Science September' and launch four projects in four weeks. These were all base don the new system and it worked out very well.
BBC Stargazing returned and in Jan 2013 we launched Planet Four, built specially for the show.
AT writing, the latest project was Worm Watch Labs, the 27th project.

%% As of September 2013, the most recent projects were \emph{Worm Watch
%%   Lab}, launched in June 2013, $24$th project to launch

% First scientific paper from Zooniverse results

Table \ref{project-summary} lists all of the Zooniverse projects,
launch dates, sizes of data sets and classifications as of September
2013.  Figure $Y$, likewise, illustrates the growth of the project
from June 2007 until September 2013.

The complete list of scientific papers that have resulted from
Zooniverse are compiled on the Zooniverse web site\footnote{Zooniverse
  Publications - \url{https://www.zooniverse.org/publications}}.

% Terminology of Zooniverse 
% See blog post by arfon > 

\begin{table*}
\begin{center}
\caption{Summary of Zooniverse projects past and present.
Notes: The 1.68 million assets in the various Galaxy Zoo projects are not unique, since galaxies in GZ1 were used in subsequent projects.}
\begin{tabular}{lcllclll}
\hline
Project & Status & URL & Launch & Category & Logged-In & Assets & Interface \\
Name &  &  & Date &  & Users &  &  Type \\
\hline
\hline
Galaxy Zoo & Retired & zoo1.galaxyzoo.org & 11 Jul 2007 & Space & 165,000 & 890,000 & Classification \\
\hline
Galaxy Zoo 2 & Retired & zoo2.galaxyzoo.org & ?? ??? 2009 & Space & XX,XXX & 304.122 & Classification \\
Galaxy Zoo Mergers & Retired & mergers.galaxyzoo.org & 23 Nov 2009 & Space & 20,588 & 58,956 & Classification \\
\hline
Solar Stormwatch & Active & solarstormwatch.com & 22 Feb 2010 & Space & 65,971 & YY,YYY & Classification/Marking \\
Galaxy Zoo Supernova & Retired & supernova.galaxyzoo.org & 26 Mar 2010 & Space & 37,150 & 76,376 & Classification \\
Galaxy Zoo: Hubble & Retired & zoo3.galaxyzoo.org & 17 Apr 2010 & Space & XX,XXX & ~200,000 & Classification \\
Moon Zoo & Active & moonzoo.org & 11 May 2010 & Space & 121,251 & 435,314 & Marking \\
Old Weather & Active & oldweather.org & 12 Oct 2010 & Climate & 32,076 & YY,YYY & Transcription \\
The Milkyway Project & Active & milkywayproject.org & 07 Dec 2010 & Space & 57,675 & 35,695 & Marking \\
Planet Hunters & Active & planethunters.org & 16 Dec 2010 & Space & 167,354 & 3,063,759 & Type \\
\hline
Ancient Lives & Active & ancientlives.org & 25 Jul 2011 & Humanities & 24,983 & 153,885 & Transcription \\
Ice Hunters & Retired & icehunters.org & 09 Aug 2011 & Space & 15,276 & YY,YYY & Classification/Marking \\
NEEMO & Active & neemo.zooniverse.org & 15 Oct 2011 & Space & X,XXX & YY,YYY & Classification \\
Whale FM & Active & whale.fm & 29 Nov 2011 & Nature & 2,150 & 15,531 & Classification \\
\hline
SETI Live & Active & setilive.org & 29 Feb 2012 & Space & 63,609 & YY,YYY & Type \\
Galaxy Zoo 4 & Active & galaxyzoo.org & 11 Sep 2012 & Space & 48,550 & 390,907 & Classification \\
Seafloor Explorer & Active & seafloorexplorer.org & 13 Sep 2012 & Nature & 14,099 & 123,077 & Marking \\
Cyclone Center & Active & cyclonecenter.org & 27 Sep 2012 & Climate & 4,767 & 196,638 & Classification \\
Bat Detective & Active & batdetective.org & 02 Oct 2012 & Nature & 1,580 & 582,203 & Classification \\
Cell Slider & Active & cellslider.net & 23 Oct 2012 & Biology & 13,261 & 275,702 & Classification \\
Andromeda Project & Active & andromedaproject.org & 05 Dec 2012 & Space & 5,072 & 12,425 & Marking \\
Snapshot Serengeti & Active & snapshotserengeti.org & 11 Dec 2012 & Nature & 22,173 & 1,240,727 & Classification \\
\hline
Planet Four & Active & planetfour.org & 08 Jan 2013 & Space & 34,718 & 98,920 & Marking \\
Notes from Nature & Active & notesfromnature.org & 22 Apr 2013 & Nature & 3,490 & 123,402 & Marking/Transcription \\
Space Warps & Active & spacewarps.org & 08 May 2013 & Space & 9,544 & 345,240 & Marking \\
Worm Watch Lab & Active & wormwatchlab.org & 30 Jun 2013 & Biology & 3,251 & 74,016 & Classification \\
\hline
\end{tabular}
\end{center}
\label{project-summary}
\end{table*}



% TODO (from Chris): Guide to the Social Science papers that have been written
% TODO add a guide to the Papers that contain project descriptions for each project

\section{Method}
\emph{TO BE WRITTEN LATER}
%% Four sessions were dedicate to performing a group thematic
%% analysis. Participants from the Zooniverse team included the team
%% lead, lead project manager, and two designers of the Planet Hunters
%% application.  Interviews were recorded and transcribed,

%% inductive process which led to the themes for the coding analysis.
%% These themes were used as the basis of a second round of interviews
%% focused on these themes, upon which a second coding process was
%% applied by 3 researchers

\section{Dimensions for Design}
% Themes? Observations?

\subsection{Key components} %  of a citizen science project

Despite the development of projects in very different domains of academic research, and which involve disparate user interactions, the core requirements for a successful citizen science project remain stable. A `main' interface allows a \emph{user} to complete a \emph{task} when presented with a \emph{subject}. The task is typically constrained by the tools provided (e.g. answer questions from a decision tree, mark craters on an image) and when completed typically results in the presentation of the next subject to be classified. In Zooniverse projects to date, participants perform tasks individually, and are typically not given control over choice of what subjects to work on. This latter choice makes deliberate manipulation of the data difficult and also ensures that project priorities are respected (i.e. it's not just the beautifu/interesting/easy subjects that are worked on). Secondly, some tutorial elements - either as a stand-alone tutorial, as an interactive within the interface or as surrounding context - are required to acquaint users with their tasks. An additional environment for discussion is typically, but not always, provided allowing for more free-form interaction between users and other participants such as scientists, developers and for which work goes beyond the initial task. Extra tools for manipulated subjects or for exploring contextual information may be provided in association with these discussion environments. 

A community of participants is implicitly necessary for a citizen science project; crowdsourcing without a crowd is a perhaps unsolvable problem. However, levels of participation and involvement in the community will naturally vary; Zooniverse projects typically receive a substantial number of their classifications from users who will never return. A successful citizen science project will thus be designed to both enable these short-term participants to make useful contributions while still fostering longer-term engagement. This need also illustrates the requirement in most cases to assess the ability or accuracy of participants; many projects thus incorporate `tests', either explicitly or more commonly by including simulated or expert-classified subjects in the workflow presented to users. 

%% from emax --- 
% attracting new users' attention
%   + engaging users in general
% getting noobs up to competence
% retaining experienced + most valuable participants
% expanding the user base
% dealing with new data ~
% transferring interest to other projects
% distilling knowledge from contributions : (amalgamating responses into thing)
% selecting the projects - understanding what can be turned into a good Citizen Science project

% almost always needs a discussion space

\subsection{Discussion forums}

The original Zooniverse project, \emph{Galaxy Zoo 1}, launched as a standalone site without a discussion environment, but a forum was added soon after launch\footnote{The original forum is still active at www.galaxyzooforum.org}, primarily in response to the number of emails received by the team. The forum unexpectedly became a vital part of the scientific machinery supported by Galaxy Zoo, with users making use of the site to identify, discuss and advocate for serendipitous discoveries. The canonical example is the object now known as Hanny's Voorwerp \citep{voorwerp}, a galaxy-scale glowing gas cloud which turned out to have been ionized by activity associated with neighboring galaxy's rapidly feeding black hole. In this case the discovery was reported on the forum, but the follow-up work was carried out by the science team themselves. In other examples, though, much more sophisticated behaviour was seen. The discovery of the Galaxy Zoo Peas \citep{Peas}, for example, saw a group of volunteers who had identified the presence of small, round and green objects (hence the name) in the background of some images work together to download and explore metadata on these objects, to write database queries and even, eventually, to create their own citizen science site to assist in further classification of these intriguing objects. The peas turned out to be a new class of galaxy, the most efficient stellar factories in the local Universe, and remain the subject of vigorous debate in the professional astronomical community. 

Structured collaboration also proved effective. Galaxy Zoo science team member Bill Keel from the University of Alabama was able to spend time on the forum, and asked for help with searches for objects such as galaxies which appear to overlap \citep{overlap} (distant galaxies used to illuminate foreground ones can be used to probe the dust content of systems, a matter of some importance to astronomers). This work was also successful, but required Keel to spend substantial time as part of the forum community, something that other science team members were unwilling or unable to do. Furthermore, by 2009 the forum had become less popular; the percentage of Galaxy Zoo users posting on what was a standard Simple Machines forum was very low (less than two percent) and with more than 500,000 posts in more than 10,000 topics it had become difficult to navigate. While it still served the need of a core constituency, it was not engaging the majority of Galaxy Zoo classifiers in science. The low participation rates were mirrored in other Zooniverse projects such as Moon Zoo and Old Weather, although the latter in particular provided a core community with a space to discuss the historical aspects of the project in particular.

Starting from the launch of Planet Hunters in 2010, Zooniverse projects have made use of a custom discussion environment known as `Talk'. It was designed to enable links to be made easily between classification and discussion and to allow science team members as well as advanced volunteers to quickly notice when users were talking about discoveries of potential interest. To understand the design drivers for talk's system of object-orientated discussion, consider the case of a new classifier who had spotted a `pea' in a Galaxy Zoo image. Even if they were to move to the forum, they would have been unable to search for discussion of their or similar objects unless they made the same mental leap to think of a small round celestial object as a kind of vegetable. If they found the right thread, they would have had to upload their image to a linear discussion which might well be in the middle of more detailed analysis. In Talk, by contrast, a single click after classification invites discussion, and lands on a page dedicated to discussing the object that's just been classified. Our putative pea-finder would immediately be able to see what had already been said about this system, and - were it already to be tagged as a 'pea' - to click on 'pea' and realize that there was a much broader conversation going on. This model has been broadly successful, with most of the discoveries made by the Planet Hunters project (including PH1b, the first planet in a system with four stars) coming from discussion between users who found interesting things in the interface and a community of more advanced workers who were able to help the science team follow-up on those discoveries. 

Different prompts to talk.

Galaxy Zoo Peas Corp and Hanny's Voorwerp.
Created Talk and then found cool planets in PH, Yellowballs tag in MWP, Convict Worm in Seafloor

% \subsection{Forums}
%% Forums -> 
%% Introduction of Talk 1.0 -> 
%% Introduction of Talk 2.0
%% Organisation and Fragmentation 
%% Top level organisation, moderators, scientists and how this has changed and impacted 

%% In a study investigating the effect of instructors on forum participation, Mazzolini and Maddison found that instructors who posted frequently on a forum on average produced shorter discussion threads \cite{mazzolini2003sage}. In the Zooniverse environment the participation of `moderators' and `scientists' could be hindering the discussion flow between general science citizens. Furthermore there was a negative correlation between instructor initiated conversations and participation, especially in the advanced units\cite{mazzolini2003sage}. 

%% @Neal does this support your findings so far?

%% Success stories: examples of super-moderators who externally test
%% contributions and distill them for the scientists
%% (why only in certain apps and not others?)
%% (how do super moderators affect the community // roles played)


\subsection{Tutorial}

Video tutorials don't work. People do them but then leave (SSW).
Compulsory training laos results in a large bounce rate (Moon Zoo)
With Whale FM and Ancient Lives we used tutorials that were interactive and overlaid on the interface - seemed to work well.
Then we moved on to the next logical step, which is guiding a user through a classification by annotating the interface as they go along. That's where we are now.

% when we turned on compulsory tutorial -- bounce rate went up

\subsection{Interface Design}

% Does good design improve participation?  Much evidence has suggested that \emph{good design builds trust}, by improving people's subjective appraisal of web sites and applications.  

%% Examples of surface level/aesthetic re-design >  (and perceived effects)
MoonZoo : redesigned December 2011 - re-designed the web site around the interface, interface had a slight modifications, and perceived effect? % look to see if there was any effect? >> not as far as we can see

%% Examples of thematic redesign encompassing addition of narrative context (and percevied effect) > 
How to measure engagement
  how much time spent on the site
  people stick over a minute, (2-3 mins, high is 20)
  (moderators of old weather)?
  % TODO: can we say anything about the effect of the narrative in Old Weather? 
  % evidence in the discussion forums ? 
  %   > i wonder why the handwriting has changed?
  %   > most keen to help each other out -- 

%% Examples of interface affordance design (and perceived effect)
GZ / GZ2 - so it was turned into a deicsion tree
  what kind of galaxy, 6 buttons -> decision tree
% much better data, and it decreased participation

% I DONT KNOW button - don't have one (blog post)

% Surface design doens't matter

Good design builds trust. There is implicit trust in a website that matches the expectations of web users. Online interfaces are judged (consciously or otherwise) in comparison to the pages that a user has recently visited (often only seconds previously) and pages they regular visit. A rising tide of improving visual aesthetic has resulted from a deacde-long virtuous cycle of modern websites attempting to keep pace with the standards of more sophisticated users.

A well-designed website creates trust in new users and comfort in returning visitors. It also encourages sharing by others.

The trajectory of design at the Zooniverse changed dramatically in 2010 with Old Weather, the Milky Way Project and Planet Hunters. % 

Ice Hunters is the exception: a project widely derided for its design, that was very popular and encouraged people to make thousands of classifcations. Could this be an indication that ease of use and accessibility (the goal of Ice Hunters wa sclear and easy to understand) trumps design, at least in the short term.

\subsection{Sustaining Engagement}

% Feedback - how many have I done? (how many eggs vs how many Xs)
% Boring project problem - Subject selection - a constraint that led to development and technological change
% Old Weather : Adding context

\subsubsection{Effect of speed/latency}

%% did they create rules - placing limitations to achieving the goal -> make a challenge
%% what feedback system did they use and what was the overall goal

%% icebergs/seal hunting : people really wanted a seal. :( 
%% audio :: 
%% posters vs 

%% adding context: old weather
%% achieving flow state

\subsection{Interaction Flow}
% jump straight to action, no introductory video
% start with 1 GS, interleave gold standards w/ real things


\subsection{Social Engagement}
% favouriting, tweeting, blog posts

\subsection{Outreach}
% wording, how when to spam participants to get them to come back

\subsection{Launching}
% Andromeda and Snapshot Serengeti as case studies

\section{Discussion}

\subsection{Common Myths of Citizen Science}
\subsection{$D$ Myths of Designing for Citizen-Science}
\subsection{Myth $X$: Putting new users through a ``tutorial'' is a good idea}
\subsection{Myth $Y$: Gameification keeps people motivated}
%% to gameify or not?
McGonigal identified four defining elements of a game: a goal, rules, feedback system and voluntary participation. Other features such as leader boards, badges, the `winning' sensation are all used to reinforce these core concepts but do not create a game environment in their own right \cite{mcgonigal2011reality}. To further this gameplay is a state which encourages an optimistic outlook on personal capabilities, partnered with `invigorating rush of activity' \cite{mcgonigal2011reality}. These concepts would support a science citizen, creating a gameplay state to highly motivate and encourage them to undertake difficult challenges. 
\subsection{Myth $Z$: Participants become domain experts}
\subsection{Myth $Z$: Beautiful pictures are necessary}
\subsection{Myth $Z$: Moderator involvement encourages discussion}

\subsection{Themes We See Support For}
\subsection{Design Builds Trust}
\subsection{Scaling/latency and practical deployment constraints during design}
\subsection{What makes a good citizen science project?}

\subsection{Comparison to Other Systems}
% Comparison with Jeremy Bentham project - failed in the sense that required more energy than was gained out of it, 20 people at the end
% YourPaintings
% Be A Martian

\subsection{Engagement: Interestingness x Difficulty x Context}


% \subsection{Unsolved mysteries}

\section{Related work}

% anyone elses' citizen science systems
\emph{Connect related work here with FoldIt, etc}

\section{Conclusion}

\section{Acknowledgments}
Acknowledgments omitted for blind review.

\balance

%% The Zooniverse framework team has derived significant has
%% been successively refined and scaled as the variety of tasks and
%% number of participants have increased.  At its current state,
%% currently having launched $X$ distinct applications for $Y$ scientific
%% domains, including astronomy, zoology, cell and marine biology,
%% archaeology and paleontology.  This platform represents a unique\cite{moore2011facebooking}


%%  These
%% applications, though separate, have been built on top 

%% The experiences from the first were used to derive design goals for
%% the next,

%% The contributions of the 
%% We identify key design challenges

%% especially as the best practices for designing citizen science systems
%% has not yet emerged.  Among the many design challenges include, being
%% able to appeal to participants with an extremely wide range of
%% expertise, ranging from no knowledge of the field to significant
%% background and interest.  Participants naturally feature a diversity
%% of natural competencies, which is manifested in some people being
%% simply much more adept at some tasks than others. Second, people have
%% many different reasons for engaging with citizen science projects, and
%% to sustain engagement, these platforms must appeal to, and engage
%% these different motivating reasons. Finally, there are a large variety
%% of issues pertaining to individual retention, well as supporting
%% various degrees of engagement -- from the ``sunday scientist'' to the
%% ``scienceoholic''.


%% The purpose of this examination of Zooniverse is to both to document
%% the experience gained from launches and iterations of the various
%% applications, comparing these experiences against previously
%% documented in other citizen-science projects.  The observations derive
%% from a lateral examination of the

%% The path from its first experimental app, Galaxy Zoo, to the more than
%% twenty different projects that have launched on the Zooniverse project
%% required generalising the findings from the first project to different
%% kinds of tasks in other scientific domains.

%%  naturally Participants come from a wide
%% audience % with a massive variety of backgrounds and competencies,
%% such systems interface down to the workflow of how participants' input
%% is collated, verified, and provided as feedback to the participants,
%% along with the nature and kind(s) of affordances provided for
%% communicating and discussing remains challenigng

%% interfaces that have
%% appropriate affordances, the and features remains challenging, due
%% to the wide number of design considerations that mustbe taken
%% jointly into account.

%% Wide variety of expertise

% \section{Background: Brief History of Zooniverse}

% \emph{For the CSCW readers, outline the history of the development of the system
% including a detailed description}

% \section{Observations through iterations}

% \emph{I was thinking put key design observations here relating to how to cross-domain
% citizen science}




% If you want to use smaller typesetting for the reference list,
% uncomment the following line:
% \small
\bibliographystyle{acm-sigchi}
\bibliography{zooniverse-history}
\end{document}

%% from crw04
%% \begin{algorithm}[tb]
%%   \caption{Overview of our general negotiation process, which is common to all of our strategies.  Let $o_\text{own}$ and $o_\text{opp}$ represent our own and the opponent's latest offers, respectively. $t_c$ is the current time and $u_\tau$ is the aspiration level at time $t_c$.}\label{alg:generic-overview}
%%   \begin{algorithmic}
%%     \FOR{$t_c \in [0,1]$}
%%     \STATE $o_\text{opp} \Leftarrow $ {\sc ReceiveOffer}()
%%     \STATE $u_\tau \Leftarrow $ {\sc SetAspirationLevel}($o_\text{opp}, t_c$)
%%     \IF{{\sc GetUtility}($o_\text{opp}, t_c$) $\geq u_\tau$}
%%     \STATE {\sc AcceptOffer}($o_\text{opp}$)
%%     \RETURN
%%     \ENDIF
%%     \STATE $o_\text{own} \Leftarrow $ {\sc GenerateOffer}($u_\tau$)
%%     \STATE {\sc ProposeOffer}($o_\text{own}$)
%%     \ENDFOR
%%     \end{algorithmic}
%% \end{algorithm}

%%  LocalWords:  artefacts HCI artefact Dropbox Skydrive Google PDF
%%  LocalWords:  LaTeX versioning throughs interactional CDSSes UI LD
%%  LocalWords:  bioinformaticians iPad iCloud iCal favour favourite
%%  LocalWords:  microformats picoformats WebDAV situ VCS scm priori
%%  LocalWords:  Powerpoint CB's CBs each's bulleted parseable OTs
%%  LocalWords:  sub-schemas pre Dourish XLSX csv PPTX PPT ICS CalDAV
%%  LocalWords:  RSS VCF XSLT XLST CSS Dojo PNG
