notes and todos

different systems have different goals
some are to educate,
others are to make science happen.
 => didn't even explain what a galaxy was.

science team inundated by emails (15,000) before 
 4 person team

did they answer their research 1qustions? did they get papers out of it?
  users are intrinsically motivated to contribute to science
  so the team ``owe it to the users'' t9o get the science out there''


% this is a sad buffer where things that were written go to die >>>>>>>>>>>>>>>>>>>>>>>>>>>

% We then discuss the implications of these findings on the design of these forums, and highlight the factors that led to 

% Central to the process of all of these discoveries were the discussion forums, originally principally the Galaxy Zoo Forum, and later \emph{Zooniverse Talk}, described later.  Although there was nothing unique about the set-up or organisation, when paired with the projects, these forums provided both the conduit and context for collaborative discovery, with sufficient flexibility to allow participants to both field each others' questions, and coordinate on more difficult cases, while also socialising and sharing subjects of interest and beauty.

% The first Zooniverse Project, \emph{Galaxy Zoo}, launched as a standalone site without a discussion environment, as the tasks themselves were perceived as the primary method of transforming participation into scientific knowledge (illustrated by the `top path' of Figure \ref{fig:twopaths}). However, when a forum was soon added to handle the flood of questions received by the science team about common artefacts and confusions (such as camera glitches, streaks caused by satellite trails, or problems with the filters) \footnote{The original forum is still active at \url{www.galaxyzooforum.org}}, it became a vital part of the system.  

% Beyond achieving the goals set out to take care of simple questions from new users, the forums also became a place by which users could share and discuss their favourite subjects they encountered, and ultimately, to perform collaborative sensemaking, by sharing, discussing and collecting evidence about unusual subjects that were spotted.  Less than a month after the GalaxyZoo Forum was launched, the citizen participant responsible for the ``Voorwerp'' discovery posted her message that would ultimately mark the project's first citizen-led discovery, of which several more followed.  This second, serendipitous ``path to science'' thus became a major focus for the Zooniverse team, illustrated as the bottom path of Figure \ref{fig:twopaths}. 

% In the following sections, we briefly describe four examples of citizen-initiated investigations, focusing on the ways that citizens contributed to each.  We then discuss the implications of these findings on the design of these forums, and highlight the factors that led to \emph{Talk}, the next iteration of the forum design.

%Therefore, by simply establishing an environment for a community to grow around a common set of tasks, the forums enabled community-driven serendipitous discovery. Here, we provide focused vignettes of four notable citizen-led initiatives, including three confirmed discoveries resulting from them, in order to convey the ways the discussion forums were used.  We then discuss the ways that the Zooniverse team refined the design of the discussion forums to better support distributed collaborative discovery-making.

% Ultimately, this process sparked the discovery of several previously unknown species, planets and phenomena, that have proven thus far among Zooniverse's most significant findings.  Thus, the Zoonivese team see two major ``Paths to Science'', illustrated in Figure \ref{fig:twopaths}. We briefly describe four such phenomena here, highlighting the unique aspects of each discovery, and the roles of the individual parties in each.
