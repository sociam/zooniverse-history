notes and todos

different systems have different goals
some are to educate,
others are to make science happen.
 => didn't even explain what a galaxy was.

science team inundated by emails (15,000) before 
 4 person team

did they answer their research 1qustions? did they get papers out of it?
  users are intrinsically motivated to contribute to science
  so the team ``owe it to the users'' t9o get the science out there''


% this is a sad buffer where things that were written go to die >>>>>>>>>>>>>>>>>>>>>>>>>>>

% We then discuss the implications of these findings on the design of these forums, and highlight the factors that led to 

% Central to the process of all of these discoveries were the discussion forums, originally principally the Galaxy Zoo Forum, and later \emph{Zooniverse Talk}, described later.  Although there was nothing unique about the set-up or organisation, when paired with the projects, these forums provided both the conduit and context for collaborative discovery, with sufficient flexibility to allow participants to both field each others' questions, and coordinate on more difficult cases, while also socialising and sharing subjects of interest and beauty.

% The first Zooniverse Project, \emph{Galaxy Zoo}, launched as a standalone site without a discussion environment, as the tasks themselves were perceived as the primary method of transforming participation into scientific knowledge (illustrated by the `top path' of Figure \ref{fig:twopaths}). However, when a forum was soon added to handle the flood of questions received by the science team about common artefacts and confusions (such as camera glitches, streaks caused by satellite trails, or problems with the filters) \footnote{The original forum is still active at \url{www.galaxyzooforum.org}}, it became a vital part of the system.  

% Beyond achieving the goals set out to take care of simple questions from new users, the forums also became a place by which users could share and discuss their favourite subjects they encountered, and ultimately, to perform collaborative sensemaking, by sharing, discussing and collecting evidence about unusual subjects that were spotted.  Less than a month after the GalaxyZoo Forum was launched, the citizen participant responsible for the ``Voorwerp'' discovery posted her message that would ultimately mark the project's first citizen-led discovery, of which several more followed.  This second, serendipitous ``path to science'' thus became a major focus for the Zooniverse team, illustrated as the bottom path of Figure \ref{fig:twopaths}. 

% In the following sections, we briefly describe four examples of citizen-initiated investigations, focusing on the ways that citizens contributed to each.  We then discuss the implications of these findings on the design of these forums, and highlight the factors that led to \emph{Talk}, the next iteration of the forum design.

%Therefore, by simply establishing an environment for a community to grow around a common set of tasks, the forums enabled community-driven serendipitous discovery. Here, we provide focused vignettes of four notable citizen-led initiatives, including three confirmed discoveries resulting from them, in order to convey the ways the discussion forums were used.  We then discuss the ways that the Zooniverse team refined the design of the discussion forums to better support distributed collaborative discovery-making.

% Ultimately, this process sparked the discovery of several previously unknown species, planets and phenomena, that have proven thus far among Zooniverse's most significant findings.  Thus, the Zoonivese team see two major ``Paths to Science'', illustrated in Figure \ref{fig:twopaths}. We briefly describe four such phenomena here, highlighting the unique aspects of each discovery, and the roles of the individual parties in each.

% Discussion bunnies >>

% \subsection{Common Myths of Citizen Science}
% \subsubsection{Myth $X$: Putting new users through a `tutorial` is a good idea}
% \subsubsection{Myth $X$: Experienced / advanced users perform better than new users}
% We have seen no support for this hypothesis; in fact an examination of projects ($X$ etc) demonstrated no correlation between duration of use and user performance \cite{simpson2013dynamic}. 
% However, there was evidence that the ways that people get better at using the interface, and that they understand what they are doing with the interface better; for example, with Snapshot Serengeti experience users generally migrate from using the more laborious decision tree interface for describing the animal's features to directly identifying the class and species of animal.

% \subsubsection{Myth $Y$: Citizen science projects have to be `gameified'}
% McGonigal identified four defining elements of a game: a goal, rules, feedback system and voluntary participation. Other features such as leader boards, badges, the `winning' sensation are all used to reinforce these core concepts but do not create a game environment in their own right \cite{mcgonigal2011reality}. To further this gameplay is a state which encourages an optimistic outlook on personal capabilities, partnered with `invigorating rush of activity' \cite{mcgonigal2011reality}. These concepts would support a science citizen, creating a gameplay state to highly motivate and encourage them to undertake difficult challenges. 

% %% The team observed that it took a considerable amount of effort to gameify projects and no net gain occurred over simply
% %% stating "x is really important/beautiful/whatever and we want to study it/save it, please help." Also, it appears that 
% %% for a game to be 'fun' it required an activity which significantly slowed the classification rate, such as in the 'ash game'
% %% that Chris and Rob mentioned while in Oxford. Instead, they opted to use Obfuscated Gameification, where aspects of gameification
% %% were used, such as leaderboards and badges, to encourage participation, without having overt gameification. In addition, users
% %% contribute because of intrinsic motivations such as "I really want to help with science" and their views of what constitutes 
% %% science and what satisfies these intrinsic motivations conflict with gameified projects (? - mentioned by people here but not heard from the team)



% \subsubsection{Myth $Y$: I should include a 'Don't Know' Button}
% % Serengeti blog post - "http://blog.snapshotserengeti.org/2012/12/14/we-need-an-i-dont-know-button/"
% % Needs of science/admin team vs desires of users - not having button far more advantageous for science team and can give tag
% % subjects even though subject may not be clear - e.g, evidently something small even though not sure what.
% % However, talk discussions show that this can lead to misuse of buttons - "nothing here" when there clearly is.
% \subsubsection{Myth $Z$: Participants become domain experts}
% % Related to a previous myth, in that users don't get more accurate with their classifications, so any expertise gain is just as likely to include 'false' expertise - that is, users might learn something about certain galaxies, while simultaneously learning fallacies about other galaxies, in which case they can hardly be said to be experts.

% \subsubsection{Myth $Z$: Beautiful pictures are necessary}
% % Planet Hunters popular, but lacks pictures - In fact uses graphs, which could hardly be called beautiful
% \subsubsection{Myth $Z$: Non-pictorial data are harder to understand}
% % Planet Hunters uses graphs but users do not need to understand the whole graph to classify - Only to identify transit features
% \subsubsection{Myth $Z$: Images don't have to be beautiful and graphs don't have to be scary)}
% There was an initial fear during the development of Planet Hunters that showing participants the light curves as infographics would result in less participation (either by turning participants away from the task) or would simply be unable to perform the task because of being unable to understand what the data represented.  Planet Hunters has been shown, however, just the opposite, and is one of the most successful apps overall.  
% \subsubsection{Myth $Z$: The Data Aren't 'Good Enough'}
% \subsubsection{Myth $Z$: If You Build It, They Will Come}

% It is a common misconception that websites are busy by default. It should go without saying that publicity and attention are required for people to find an online citizen science project. Networks and communities exist online to allow people to discover and share online projects. The Zooniverse was one of the first organisations in this field and has grown a substantial network of people around it (\~860,000 at time of writing).

skdjfdf

% However above simple awareness, any creator of a citizen science project should be aware of the approximate effort required. Most Zooniverse sites enlist the help of tens of thousands of online volunteers. Thus when projects are designed, consideration should be given to the scale of the endeavour being attempted. Taking into account a reasonable expectation of web traffic and the effort any person may put in is difficult (how long is a piece of string?) but important for establishing a project with a realistic end goal within the required time. Simply producing a citizen science website does not guarantee popularity -- or more importantly scientific completion of the intended task.

% % Zooniverse projects draw a substantial number of contributors from the pre-existing community - See Andromeda Project mentioned above, completed in sixteen days due to pre-existing community being informed by newsletter. 
% % Without this community, would have taken much longer. Difficult to form community without 'luck' - Sky at Night and media plugs for various projects, for example, more down to who is on the team. Certainly not users flocking to Zooniverse just because it exists.

% \subsubsection{Myth $Z$: Moderator involvement encourages discussion}
% % comments from Rob and Chris?

% Studies conducted in the educational sector have discovered that contrary to popular belief, instructors (i.e. experts) who contribute often to discussions actually decreased student posts \cite{zydney2012creating}. However, Mazzolini and Maddison propose that this reduction could be a result of more efficient discussion and understanding \cite{mazzolini2007jump}. 

% In a forum context, an interactive post is one which responds or replies to another's message, whereas a participation is the number or length of a post \cite{schrire2006knowledge}. 
% Schrire discovered two distinctive interaction patterns: instructor-centered, messages predominantly responded to the post by the instructor and synergistic, student-student collaboration \cite{schrire2006knowledge}. 

% \subsection{What makes a good citizen science project?}
% \subsubsection{Design Builds Trust}
% % \subsection{Scaling/latency and practical deployment constraints during design}

% \subsection{Comparison to Other Systems}
% % Comparison with Jeremy Bentham project - failed in the sense that required more energy than was gained out of it, 20 people at the end
% % YourPaintings
% % Be A Martian
% % FoldIt

% \subsection{A Heuristic Framework for Citizen Science App Designers}

%% In order to put the observations described above into a more concise, easily communicated and form, we assembled common themes into a multidimensional framework of design heuristics for citizen science system designers, comprising 6 constructs, described below.  Each construct is meant to address a key dimension of the necessary components described earlier, using the observations discussed in findings. The purpose of such a framework is intended both as an artefact for discussion and refinement, and potential practical use by designers seeking to apply insights from the Zooniverse team  currently made by the Zooniverse team.

% attracting new users' attention / expanding the user base
% identifying 'good' projects - understanding what can be turned into a good Citizen Science project
% contextualising the project
%   interestingness/difficulty/conceptual+contextual+narrative threads that tie the tasks together/understand the point of it
%
% performing the task - 
%   - tutorials
%   - elicitation/task vtrtvgtvvt4interface considerations // wide open, specific labeling, decision trees
%   - levels of design and their considerations (aesthetics, challenge, interestingness)
% discussion and collaboration - 
% retaining experienced + most valuable participants
% dealing with new data ~
% distilling knowledge from contributions : (amalgamating responses into thing)

% other/misc
%   transferring interest to other projects
% almost always needs a discussion space

