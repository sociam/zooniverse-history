\begin{table*}
\begin{center}
\small
\begin{tabular}{lcllclll}
\hline
Project & Status & URL & Launch & Domain & Logged-In & Subjects & Task \\
Name &  &  & Date &  & Users &  &   \\
\hline
\hline
Galaxy Zoo & Retired & zoo1.galaxyzoo.org & 11 Jul 2007 & Space & 165,000 & 890,000 & Classifying \\
\hline
Galaxy Zoo 2 & Retired & zoo2.galaxyzoo.org & ?? ??? 2009 & Space & XX,XXX & 304.122 & Classifying \\
Galaxy Zoo Mergers & Retired & mergers.galaxyzoo.org & 23 Nov 2009 & Space & 20,588 & 58,956 & Classifying \\
\hline
Solar Stormwatch & Active & solarstormwatch.com & 22 Feb 2010 & Space & 65,971 & YY,YYY & Classifying/Marking \\
Galaxy Zoo Supernova & Retired & supernova.galaxyzoo.org & 26 Mar 2010 & Space & 37,150 & 76,376 & Classifying \\
Galaxy Zoo: Hubble & Retired & zoo3.galaxyzoo.org & 17 Apr 2010 & Space & XX,XXX & ~200,000 & Classifying \\
Moon Zoo & Active & moonzoo.org & 11 May 2010 & Space & 121,251 & 435,314 & Marking \\
Old Weather & Active & oldweather.org & 12 Oct 2010 & Climate & 32,076 & YY,YYY & Transcribing \\
The Milkyway Project & Active & milkywayproject.org & 07 Dec 2010 & Space & 57,675 & 35,695 & Marking \\
Planet Hunters & Active & planethunters.org & 16 Dec 2010 & Space & 167,354 & 3,063,759 & Marking \\
\hline
Ancient Lives & Active & ancientlives.org & 25 Jul 2011 & Humanities & 24,983 & 153,885 & Transcribing \\
Ice Hunters & Retired & icehunters.org & 09 Aug 2011 & Space & 15,276 & YY,YYY & Classifying/Marking \\
NEEMO & Active & neemo.zooniverse.org & 15 Oct 2011 & Space & X,XXX & YY,YYY & Classifying \\
Whale FM & Active & whale.fm & 29 Nov 2011 & Nature & 2,150 & 15,531 & Classifying \\
\hline
SETI Live & Active & setilive.org & 29 Feb 2012 & Space & 63,609 & YY,YYY & Classifying \\
Galaxy Zoo 4 & Active & galaxyzoo.org & 11 Sep 2012 & Space & 48,550 & 390,907 & Classifying \\
Seafloor Explorer & Active & seafloorexplorer.org & 13 Sep 2012 & Nature & 14,099 & 123,077 & Marking \\
Cyclone Center & Active & cyclonecenter.org & 27 Sep 2012 & Climate & 4,767 & 196,638 & Classifying \\
Bat Detective & Active & batdetective.org & 02 Oct 2012 & Nature & 1,580 & 582,203 & Classifying \\
Cell Slider & Active & cellslider.net & 23 Oct 2012 & Biology & 13,261 & 275,702 & Classifying \\
Andromeda Project & Active & andromedaproject.org & 05 Dec 2012 & Space & 5,072 & 12,425 & Marking \\
Snapshot Serengeti & Active & snapshotserengeti.org & 11 Dec 2012 & Nature & 22,173 & 1,240,727 & Classifying \\
\hline
Planet Four & Active & planetfour.org & 08 Jan 2013 & Space & 34,718 & 98,920 & Marking \\
Notes from Nature & Active & notesfromnature.org & 22 Apr 2013 & Nature & 3,490 & 123,402 & Marking/Transcribing \\
Space Warps & Active & spacewarps.org & 08 May 2013 & Space & 9,544 & 345,240 & Marking \\
Worm Watch Lab & Active & wormwatchlab.org & 30 Jun 2013 & Biology & 3,251 & 74,016 & Classifying \\
\hline
\end{tabular}
\normalsize
\label{table:project-summary}
\caption{Summary of Zooniverse projects past and present, including each projects status as of September 2013.  The 1.68 million assets in the various Galaxy Zoo projects are not unique, since galaxies in GZ1 were used in subsequent projects.}
\end{center}
\end{table*}

% According to neemo.zooniverse.org, NEEMO had 47,943 user validations, though this may differ from "logged-in users." Particularly because two figures are given - 485 volunteers joined the mission.

Citizen-science activities can be broadly divided into several core categories: data collection (e.g., \cite{zook2010volunteered}, eBird), data analysis (GIVE REFERENCES OTHER THAN ZOONIVERSE e.g. Stardust@Home), and problem-solving (e.g., FOLDIT, ETERNA). All Zooniverse projects fall into the second category. Data, that is, collections of digital artifacts such as images, video and audio recordings on a given theme, is collected by professional researchers, and Zooniverse users enrich it by identifying, classifying, marking and labeling objects in these artifacts. As documented previously by Fortson et al \cite{fortson2011galaxy}, the first Zooniverse project, \emph{Galaxy Zoo}\footnote{Galaxy Zoo - \url{www.galaxyzoo.com}}, launched July $2007$, engaged volunteers in the morphological classification of images of galaxies \cite{galaxyzoo-launch}. Reaching a user base of more then $100,000$ users by $2008$, the large amount of information created by the crowd soon resulted into scientific papers on spin statistics of spiral galaxies\cite{land2008galaxy} and the morphological characteristics of galaxies\cite{lintott2008galaxy}. The success of Galaxy Zoo generated significant interest, and it became clear that there would be more projects to follow.  The first such project was \emph{Solar Stormwatch}\footnote{Solar Stormwatch - \url{www.solarstormwatch.com}}, effectively built for the Royal Observatory of Greenwich by Zooniverse.  The first non-astronomy project, \emph{Old Weather}\footnote{Old Weather - \url{www.oldweather.org}}, launched in October $2010$ and applied volunteers to transcribe historical, hand-written weather measurements from official logs of merchant trading ships from the period of $1780$ to $1830$.  As of September $2013$, the Zooniverse family of projects has benefited from the participation of $XXX,XXX$ volunteers who have collectively submitted over $NNN,NNN,NNN$ contributions across different projects to the system.

Table \ref{table:project-summary} lists all the Zooniverse projects by launch date, along with their current status, their Web site location, size of data set, and task type, at the time of writing. \emph{Retired} projects are projects where sufficient input has already been acquired from the crowd for each item of the data collection and thus are not open for contributions any longer. \emph{Active} projects are those in which this level has not yet been achieved. \emph{Classifying} tasks ask users to identify the presence, types, and potentially number of things visible in each media item they see.  An example of such a task is identifying and counting the number of animals visible in a particular automatically-taken photo, taken by motion-detecting camera in \emph{Snapshot Serengeti} project. \emph{Marking} tasks involve the additional step of indicating where the particular object was found, for example, identifying the location of organisms in the undersea images of \emph{Seafloor Explorer}.  Finally, \emph{transcribing} refers to reading, often deciphering difficult-to-read, text and characters typically from old handwritten texts, such as the ship logs in \emph{Old Weather} or registers of museum specimens in \emph{Notes from Nature}. Figure 1, likewise, illustrates the growth of the project from June $2007$ until September $2013$. The complete list of scientific papers that have resulted from Zooniverse are compiled on the Zooniverse blog.\footnote{Zooniverse Publications - \url{www.zooniverse.org/publications}} The Galaxy Zoo project alone has produced $39$ scholarly publications, including a catalogue of galaxy morphologies \cite{lintott2008galaxy}, and papers on its two most famous discoveries, the ``Hanny's Voorwerp'' \cite{lintott2009galaxy} and the ``Green Peas'', a previously unknown class of galaxy \cite{cardamone2009galaxy}. Later in the paper we will briefly discuss how these prominent success stories have been enabled through specific design decisions and user engagement strategies implemented by the Zooniverse team. Pertaining to the Zooniverse projects themselves, Raddick et al conducted interview studies to understand the motivations of Galaxy Zoo project participants \cite{raddick2010galaxy}. For additional information we refer to reader to the Zooniverse Web site; each Zooniverse project has its own dedicated blog (e.g., \cite{galaxyzooblog}), while a separate, overarching blog documents news of major milestones, events, and discoveries across all Zooniverse projects to the wider community \cite{zooniverseblog}.

\subsection{Terminology and Overview of the Zooniverse Citizen-Science Platform}

In the remainder of this paper, we adopt the Zooniverse team's terminology for describing their citizen-science platform. We will briefly summarise this terminology here. The Zooniverse platform contains a set of core technology components, which can be instantiated in various forms and combinations to run specific \emph{projects} with their own scientific tasks and data sets. Such projects are proposed to the Zooniverse team by researchers, who become the project's \emph{science team}.  The science team are ultimately responsible for the project once launched; this spans the provisioning of the data to be analyzed by the crowd, but also community engagement activities. Each project represents a single, specific line of inquiry and data set, and is launched as a separate Web site, with a dedicated discussion forum,  and associated social media channels, including a blog, a Twitter account, and a Facebook page.  Volunteer participants are referred to as \emph{users}. Once signed-up they can contribute to any of the Zooniverse projects; each of these projects displays links to other projects in the Zooniverse family at the top of their Web site.

Active participation in a Zooniverse project is explicitly rewarded. Users who solve a large number of tasks or are engaging in the discussions may be appointed \emph{moderators}. They are granted special privileges in the discussion forums, as well as direct access to contact the \emph{science teams} in charge of the project. The term \emph{subject} refers to the individual assets that users are tasked with analyzing.  Subjects may come in different forms, such as images, audio files and videos. \emph{Tasks} are the particular activities which users carry out when presented with a subject, such as identifying the presence and shape of a galaxy in a set of subjects. The generic term for such a galaxy would be \emph{object} - in other words, users analyze a subject (e.g., an image) to find, classify, mark, or label objects contained by the subject. Subjects are typically organised into \emph{groups}, which represent a particular data set collection, such as those from a particular time period or source such as the Hubble telescope.  Finally, a \emph{Classification} is a single classifying, marking or transcribing action performed by a single user, upon a subject, within the context of some task, of a project.

% \subsection{Previous Zooniverse Studies}
% (( TODO : Write up a summary of the social science and systems Zooniverse papers ))